\section{Threats to Validity}
\label{sec:threats-to-validity}
This section prsents the internal validity and external validity threats to this study.

\subsection{Internal Validity}
\label{sec:threats-to-validity-internal}
\subsubsection{Code Smells are in part related to OOP}
Many of the code smells defined in the litterature as well as in this study are related to OOP. Code smells such as Too Many Attribute and Too Many Methods are directly linked to the usage of classes in a program. However, Python programs are not always written in an OOP fashion, compared to Java or C++ programs.

Furthermore, the short nature of code snippets mean that the code smells that are related to OOP might not appear as often as they would in a full program. This means that the analysis of code smells for such code snippets might not be the best way to measure code quality. \\

\subsubsection{Multiple conversations containing the same code snippet}
By manually going through some conversations between developers and ChatGPT, we realized that the AI sometimes provided the same code snippet in follow up answers. This means that code smells in a code snippet would be counted multiple times, but when integrating the code into a project, they would only appear once. This could therefore skew the results of the analysis. \\

\subsubsection{Code smells can positively impact code quality}
As presented in section \ref{sec:related-work-relation-between-code-smell-and-software-quality}, the relationship between code smells and code quality is not always negative. Some code smells can actually improve software qualities. Therefore, the presence of code smells in ChatGPT's answers might not be entirely negative in terms of code quality. Further studies would have to be performed to see what qualities are impacted by the code smells found in this study. \\

\subsection{External Validity}
\label{sec:threats-to-validity-external}

\subsubsection{Generalization of the findings}
This study only looked at the code snippets generated by ChatGPT and found on Stack Overflow for problems in Python. Therefore, the results of this study might not be generalizable to other programming languages. \\

\subsubsection{Only one source of comparison}
This study also only compared the code smells found in the code snippets generated by ChatGPT to the code smells found in the code snippets found on Stack Overflow. Therefore, the results of this study might not be generalizable to other sources of code snippets, such as documentation or tutorials. \\

\subsubsection{No-change assumption}
This study assumes that the code snippets generated by ChatGPT are not modified by the developers before being integrated into their projects. However, this is likely not the case, as changes will often be required to integrate such code in projects. Developers might therefore modify the code snippets and remove the code smell found, which would not impact the quality of their software. \\